\documentclass[paper=letter, fontsize=12pt]{scrartcl}

\usepackage[T1]{fontenc}
\usepackage[utf8]{inputenc}
\usepackage[english, spanish]{babel}
\usepackage{hyperref}
\usepackage{amsfonts,amsthm}
\usepackage{amsmath}
\usepackage{listings}
\usepackage[dvipsnames]{xcolor}
\usepackage{sectsty}
\usepackage{dirtree}
\usepackage{enumitem}

\lstdefinestyle{mystyle}{
    backgroundcolor=\color{backcolour},
    commentstyle=\color{codegreen},
    keywordstyle=\color{magenta},
    numberstyle=\tiny\color{codegray},
    stringstyle=\color{codepurple},
    basicstyle=\footnotesize,
    breakatwhitespace=false,
    breaklines=true,
    captionpos=b,
    keepspaces=true,
    numbers=left,
    numbersep=5pt,
    showspaces=false,
    showstringspaces=false,
    showtabs=false,
    tabsize=2
}

\setlength{\DTbaselineskip}{20pt}
\DTsetlength{1em}{3em}{0.1em}{1pt}{4pt}

\allsectionsfont{\raggedright\large \textit\normalfont\scshape\emph}

\title{Ejercicio Semanal 1}

\subtitle{
  Lógica Computacional, 2018-2\\
  Facultad de Ciencias, UNAM
}

\author{
  \normalsize
  Noé Salomón Hernández Sánchez\\
  \normalsize
  \texttt{\href{mailto:no.hernan@gmail.com}{no.hernan@gmail.com}}
  \and
  \normalsize
  María del Carmen Sánchez Almanza\\
  \normalsize
  \texttt{\href{mailto:carmensanchez@ciencias.unam.mx}{carmensanchez@ciencias.unam.mx}}
  \and
  \normalsize
  Albert Manuel Orozco Camacho\\
  \normalsize
  \texttt{\href{mailto:alorozco53@ciencias.unam.mx}{alorozco53@ciencias.unam.mx}}
}

\date{\today}

\begin{document}

\maketitle

\section{Objetivo}

\noindent
Que el alumno empiece a conocer y/o ejercitar sus habilidades en programación funcional. Se familiarizará,\
además, con la sintaxis y el modo de ejecución del lenguaje de programación \verb+Haskell+.\
Se trabajará con listas, recursión, tipos de datos \verb+Enum+ (recursivos) y flujos de control.

\section{Ejercicios}

\noindent
El esqueleto de código del ejercicio semanal se encuentra en \url{https://github.com/alorozco53/LabLogComp-2018-2/tree/ejsem1}.

\subsection{Listas}

\begin{enumerate}
\item Proponga una función \verb+myReverse+ que calcule la reversa de una lista.\
  No utilice la función \verb+reverse+ del \emph{Preludio}.
\item Escriba una función \verb+myTake+ que, dado un entero $k$ y una lista $l$,\
  devuelva el prefijo de tamaño $k$ de $l$. Por ejemplo,
  \begin{lstlisting}[language=Haskell]
    *EjercicioSemanal1> myTake 3 [5,6,7,10,12,34,0]
    [5,6,7]
  \end{lstlisting}
  No utilice la función \verb+take+ del \emph{Preludio}.
\item Elabore una función \verb+myCount+ que, dado un elemento de cierto tipo y una\
  lista del mismo tipo, cuente cuántas veces se repite dicho elemento en la lista.
\item Escriba una función \verb+myFreq+ que, dada una lista $l$, devuelva la lista de\
  $2$-tuplas, donde, para cada tupla, la primera entrada es un elemento de $l$ y la\
  segunda es el número de veces que dicho elemento se repite en $l$. Se otorgarán\
  $\frac{1}{4}$ de puntos extras por una implementación \emph{elegante}. Por ejemplo:
    \begin{lstlisting}[language=Haskell]
      *EjercicioSemanal1> myFreq "alonzo church"
      [('a',1),('l',1),('o',2),('n',1),('z',1),('o',2),
       (' ',1),('c',2),('h',2),('u',1),('r',1),('c',2),('h',2)]
    \end{lstlisting}
\item Dada una lista $l$ y dos enteros $i$, $j$, devuelve la sublista que empieza\
  en el $i$-ésimo elemento de $l$ y termina en el $j-1$-ésimo. Esto es, análogo del\
  \verb+list[i:j]+ de Python.
\item Dada una lista de cierto y un elemento del mismo tipo, devuelve la lista de listas
  formada por todas las sublistas que preceden y suceden las apariciones del elemento
  Por ejemplo:
  \begin{lstlisting}[language=Haskell]
    *EjercicioSemanal1> split "the quick brown fox" ' '
    ["the","quick","brown","fox"]
  \end{lstlisting}
  Análogo al list.split() de Python.
\item Considere el siguiente algoritmo de compresión de cadenas: \textit{dada una palabra de tamaño} $k$
  \textit{alfanumérica (sin espacios entre carácter), mantenga únicamente los primeros}
  $\lfloor\frac{k}{2}\rfloor$ \textit{elementos de ésta; realice esto para cada palabra (en orden)}\
  \textit{de la cadena en cuestión; finalmente, concatene todos los prefijos restantes}.\
  Proponga una función \verb+dumbCompress+ que implemente el algoritmo anterior. Por ejemplo:
  \begin{lstlisting}[language=Haskell]
    *EjercicioSemanal1> dumbCompress "la science n'a pas de patrie"
    "lscinpdpat"
  \end{lstlisting}
\end{enumerate}

\subsection{Números naturales}

\noindent
Considere el tipo de datos de números naturales
\begin{lstlisting}[language=Haskell]
  data Nat = Zero | Succ Nat  
\end{lstlisting}
para la elaboración de lo siguiente.

\begin{enumerate}[resume]
\item Escriba una función \verb+sumNat+ que sume dos números naturales.
\item Proponga una función \verb+prodNat+ que multiplique dos números naturales.
\item Elabore una función \verb+powerNat+ que eleve un número natural a la potencia de otro.
\item Escriba una función \verb+eqNat+ que decida si dos números naturales son iguales.\
  No utilice la función \verb+(==)+ auto-implementada en la declaración del tipo \verb+Nat+.
\item Dé una función \verb+greaterThan+ que, dados números naturales $n$ y $m$, decida\
  si $n$ es mayor (estricto) que $m$.
\item Escriba una función que convierta un número natural de tipo \verb+Nat+ en un entero\
  de tipo \verb+Int+.
\item Escriba una función que convierta un entero de tipo \verb+Int+ en un\
  número natural de tipo \verb+Nat+.
\item Elabore una función \verb+throwRev+ que, dado un \verb+Nat+ $n$ y una lista $l$,\
  quite los $n$ últimos elementos de $l$.
\end{enumerate}

\section{Entrega}

\noindent
La fecha de entrega es el próximo \textbf{jueves 8 de febrero de 2018} por la plataforma\
de \emph{Google Classroom} del curso y siguiendo los lineamientos del laboratorio.

\end{document}
